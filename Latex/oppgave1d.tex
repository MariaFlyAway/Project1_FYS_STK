\documentclass[11pt,a4paper,oldfontcommands]{memoir}
\usepackage[utf8]{inputenc}
\usepackage[T1]{fontenc}
\usepackage{microtype}
\usepackage[dvips]{graphicx}
\usepackage{times}
\usepackage{amsmath}
\usepackage{multirow}
\usepackage[dvipsnames]{xcolor}
\usepackage{mathtools,amsfonts,amssymb,amsthm}

\graphicspath{ {./Bilder/} }
\usepackage{wrapfig}

\usepackage[
breaklinks=true,colorlinks=true,
linkcolor=black,urlcolor=black,citecolor=black,% PRINT
bookmarks=true,bookmarksopenlevel=2]{hyperref}

\usepackage{geometry}
\geometry{
left=20mm,right=20mm, top = 25mm, bottom = 25mm}

\OnehalfSpacing

%%% CHAPTER'S STYLE
\chapterstyle{bianchi}

%%% STYLE OF SECTIONS, SUBSECTIONS, AND SUBSUBSECTIONS
\renewcommand*{\chapnamefont}{\normalfont\large\rmfamily\itshape}

%%% STYLE OF PAGES NUMBERING
\pagestyle{plain}
\makepagestyle{plain}
\makeevenfoot{plain}{}{}{\thepage}
\makeoddfoot{plain}{}{}{\thepage}
\makeevenhead{plain}{}{}{}
\makeoddhead{plain}{}{}{}

\maxsecnumdepth{subsection} % chapters, sections, and subsections are numbered
\maxtocdepth{subsection} % chapters, sections, and subsections are in the Table of Contents

%%% Theorems
\newtheorem*{theorem}{Teorem}
\newtheorem*{lemma}{Lemma}

\renewcommand{\qedsymbol}{$\square$}    % symbol proof
\renewcommand*{\proofname}{Bevis}       % name of proof

%%% Math shortcuts
\newcommand{\R}{{\mathbb R}}
\newcommand{\N}{{\mathbb N}}
\newcommand{\Z}{\mathbb Z}
\newcommand{\Q}{\mathbb Q}
\newcommand{\ra}{\rightarrow}


\title{Title}
\author{Author}
\date{\today}

\begin{document}

\begin{align*}
    E(y_i) &= E(f(x_i) + \epsilon)\\
    &= E(\beta_0 + \beta_1x_{i1} + \dots + \beta_{n-1} x_{i n-1} + \epsilon)\\
    &= E\left(\sum_{j=0}^{n-1}\beta_{j}x_{ij}\right) + E(\epsilon)\\
    &= \sum_{j=0}^{n-1}\beta_{j}x_{ij}
\end{align*}
Where we've used the linearity of expectation.\\[10pt]
%
\begin{align*}
    V(y_i) &= V(f(x_i) + \epsilon)\\
    &= V(\beta_0 + \beta_1x_{i1} + \dots + \beta_{n-1} x_{i n-1} + \epsilon)\\
    &= V(\beta_0 + \beta_1x_{i1} + \dots + \beta_{n-1} x_{i n-1}) + V(\epsilon)\\
    &= V(\epsilon)\\
    &= \sigma^2
\end{align*}
Where we've used that the variance is invariant.\\[20pt]
%
\begin{align*}
    E(\hat \beta) &= E[(\boldsymbol{X}^T \boldsymbol{X})^{-1}\boldsymbol{X}^T \boldsymbol{y}]\\
    &= (\boldsymbol{X}^T \boldsymbol{X})^{-1}\boldsymbol{X}^T E[\boldsymbol{y}]\\
    &= (\boldsymbol{X}^T \boldsymbol{X})^{-1}\boldsymbol{X}^T \boldsymbol{X \beta}\\
    &= \boldsymbol{\beta}
\end{align*}
Where we've again used the linearity of the expectation.\\[10pt]
%
\begin{align*}
    V(\hat \beta) &= V[(\boldsymbol{X}^T\boldsymbol{X})^{-1} \boldsymbol{X}^T\boldsymbol{y}]\\
    &= [(\boldsymbol{X}^T\boldsymbol{X})^{-1} \boldsymbol{X}^T] \; V[\boldsymbol{y}] \;
    [(\boldsymbol{X}^T\boldsymbol{X})^{-1} \boldsymbol{X}^T]^T\\
    &= [(\boldsymbol{X}^T\boldsymbol{X})^{-1} \boldsymbol{X}^T] \; \sigma^2 \boldsymbol{I} \;
    [\boldsymbol{X}(\boldsymbol{X}^T\boldsymbol{X})^{-1}]\\
    &= \sigma^2 [(\boldsymbol{X}^T\boldsymbol{X})^{-1} \boldsymbol{X}^T
    \boldsymbol{X}(\boldsymbol{X}^T\boldsymbol{X})^{-1}]\\
    &= \sigma^2 (\boldsymbol{X}^T\boldsymbol{X})^{-1}
\end{align*}
Where we've used that if $\boldsymbol{A}$ is a contant matrix, 
$V(\boldsymbol{AX}) = \boldsymbol{A} V(\boldsymbol{X}) \boldsymbol{A}^T$.
\textbf{e)}\\
Derivation of the bias-variance tradeoff:
\begin{align*}
    E\left[(\boldsymbol{y} - \boldsymbol{\tilde y})^2\right] &= E\left[(\boldsymbol{y} - E(\boldsymbol{\tilde y}) + E(\boldsymbol{\tilde y}) - \boldsymbol{\tilde y})^2\right]\\
    &= E\left[(\boldsymbol{y} - E(\boldsymbol{\tilde y}))^2 + 2(\boldsymbol{y} - E(\boldsymbol{\tilde y}))(E(\boldsymbol{\tilde y}) - \boldsymbol{\tilde y}) + (\boldsymbol{\tilde y} - E(\boldsymbol{\tilde y}))^2\right]\\
    &= E\left[(\boldsymbol{y} - E(\boldsymbol{\tilde y}))^2\right] + 2E\left[(\boldsymbol{y} - E(\boldsymbol{\tilde y}))(E(\boldsymbol{\tilde y}) - \boldsymbol{\tilde y})\right] + E\left[(\boldsymbol{\tilde y} - E(\boldsymbol{\tilde y}))^2\right]\\
\end{align*}
The first term in the expression above can be simplified as follows:
\begin{align*}
    E\left[(\boldsymbol{y} - E(\boldsymbol{\tilde y}))^2\right] &= E\left[(f(x) + \epsilon - E(\boldsymbol{\tilde y}))^2\right]\\
    &= E\left[(f(x)-\boldsymbol{\tilde y})^2 - 2(\epsilon(E(\boldsymbol{\tilde y}) - f(x))) + \epsilon^2\right]\\
    &= E\left[(f(x)-\boldsymbol{\tilde y})^2\right] - 2E\left[\epsilon(E(\boldsymbol{\tilde y}) - f(x))\right] + E[\epsilon^2]\\
    &= E\left[(f(x) - \boldsymbol{\tilde y})^2\right] - 2E[\epsilon]E\left[E(\boldsymbol{\tilde y}) - f(x)\right]+ \sigma^2\\
    &= E\left[(f(x)-\boldsymbol{\tilde y})^2\right] + \sigma^2
\end{align*}
If we now substitute this expression in () we get:
\begin{align*}
    E\left[(\boldsymbol{y} - \boldsymbol{\tilde y})^2\right] &= (E\left[f(x)-(\boldsymbol{\tilde y})^2\right] + \sigma^2) + 2(E(\boldsymbol{\tilde y}) - \boldsymbol{y})E\left[(\boldsymbol{\tilde y} - E(\boldsymbol{\tilde y}))\right]+V(\boldsymbol{\tilde y})\\
    &= Bias[\boldsymbol{\tilde y}]^2 + 2(E(\boldsymbol{\tilde y}) - \boldsymbol{y})(E(\boldsymbol{\tilde y}) - E(\boldsymbol{\tilde y})) + V(\tilde y) + \sigma^2\\
    &= Bias[\boldsymbol{\tilde y}]^2 + V(\tilde y) + \sigma^2
\end{align*}

\end{document}